
\documentclass[openany, a4paper, 12pt]{book}

% パッケージ
\usepackage[pdfborder={0 0 0}, colorlinks=false]{hyperref} % リンクの色なし
\usepackage{graphicx}
\usepackage{xeCJK}
\usepackage[font=small,labelfont=bf]{caption} 
\usepackage[a4paper, margin=3cm]{geometry}
\usepackage{hyperref}
\usepackage{setspace}
\usepackage{fancyhdr}
\usepackage{float}
\usepackage{fontspec}  % フォント指定に必要

% List of figures
\usepackage{tocloft}
\renewcommand{\cftfigpresnum}{Figure~}  % "Figure " を番号の前につける
\renewcommand{\cftfignumwidth}{7em}     % 幅を調整(Figure~2.1などを収める)
% List of tables
\renewcommand{\cfttabpresnum}{Table~}   % ← "Table " を番号の前に表示
\renewcommand{\cfttabnumwidth}{7em}     % ← 表示幅の調整(必要に応じて調整)

\usepackage[justification=raggedright,singlelinecheck=false]{caption} % キャプションを左揃え
\pagestyle{plain}

% 日本語と英語のタイトルを定義する
\newcommand{\etitle}{English Title}
\newcommand{\etitlejp}{日本語タイトル}

\setmainfont{Times New Roman} % 英語フォント
\setCJKmainfont{ヒラギノ明朝 ProN}  % Macでよく使える日本語フォント

% Chapterの見出しを小さく整える
\usepackage{titlesec}
\titleformat{\chapter}[display]
  {\normalfont\bfseries}
  {\fontsize{10pt}{12pt}\selectfont \chaptername\ \thechapter}
  {10pt}
  {\Huge}

% -----------------------------------------------------

% ここから本文
\begin{document}
\thispagestyle{empty}

\setmainfont{Times New Roman}

%\pagenumbering{gobble}  % ページ番号を表示しない

\vspace*{2cm}
\begin{center}
    {\LARGE \textbf{Bachelor Thesis}}\\[1em]
    Academic year 2025
\end{center}

\vspace*{4cm}

\begin{center}
    \LARGE \textbf{Thesis title}\\[0.5em]
\end{center}

\vfill

\begin{center}
    Keio University\\
    学部\\[1em]
    学籍番号\\
    名前\\[1em]
    メールアドレス
\end{center}


\setcounter{page}{1}  % ページ番号を1にリセット
\pagestyle{plain}
% 英語のアブスト
\section*{Abstract (English)}
\begin{center}
  \begin{large}
    \begin{tabular}{|p{0.97\linewidth}|}
        \hline
          \etitle \\ \hline
    \end{tabular}
  \end{large}
\end{center}

\begin{spacing}{1.1}
Type text here
\end{spacing}

% ------------------------------------

% 日本語のアブストを書く
\newpage  % ←ここでページを変える

\section*{Abstract (Japanese)}
\begin{center}
\begin{large}
\setlength{\fboxsep}{0pt} 
\setlength{\arrayrulewidth}{0.4pt} 
  \renewcommand{\arraystretch}{1.5} 
    \begin{tabular}{|p{0.97\linewidth}|}
        \hline
          \centering \textbf{\etitlejp} \\
        \hline
        \renewcommand{\arraystretch}{1.5} 
    \end{tabular}
    \end{large}
    \end{center}

\begin{spacing}{1.1}
テキストを入力


\tableofcontents
\clearpage
\listoffigures   % 図の一覧(List of Figures)
\listoftables    % 表の一覧(List of Tables)
\clearpage

\chapter{Introduction}

参考文献の挿入~\cite{yugi2014reconstruction}
参考文献の挿入~\cite{yugi2016trans}
\\ % 改行はこのようにする
\\
図を参照(Fig.~\ref{fig:Fig1})
\\
\\
リンクの挿入(\url{https://www.ims.riken.jp/labo/58/cv_j.html})
\\
\\
% -----------------------------------------------------
symbolフォントの入力方法の例:
\\
\\
$\alpha$, $\beta$, $\mu$L, 2 x $10^5$/mL, CO$_2$
% -----------------------------------------------------


\end{spacing}
\begin{figure}[htbp]
    \centering
    \includegraphics[width=1.0\textwidth]{img/Fig1.pdf}  % imgディレクトリにある図を挿入
    \caption{Figre title}
    \label{fig:Fig1}
\end{figure}
\chapter{Materials and Methods}

% -----------------------------------------------------

\section{Section title}
表を参照(Table ~\ref{tab:Table1}) 
% table 1
\begin{table}[H]
    \caption{Table title.}
    \includegraphics[width=1.0\textwidth]{img/Table1.pdf}
    \label{tab:Table1}
\end{table}

% -----------------------------------------------------

\chapter{Result}

% -----------------------------------------------------

\section{Section title}
図を参照(Fig.~\ref{fig:Fig2}A)

\begin{figure}[htbp]
    \includegraphics[width=1.0\textwidth]{img/Fig2.pdf}
    \caption[Figure title]{%
        \textbf{Figure title.}%
        \newline
        Type text here
    }\label{fig:Fig2}
\end{figure}

% -----------------------------------------------------

\chapter{Discussion}

% -----------------------------------------------------

Type text here
\chapter*{Acknowledgment} % アスタリスク*をつけることによって、目次にAcknowledgmentを表示させない様にしている

% -----------------------------------------------------

Type text here

\bibliographystyle{unsrt}
%\bibliographystyle{abbrv}
\bibliography{ref.bib}

\end{document}
